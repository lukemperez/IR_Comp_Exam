% Created 2016-08-10 Wed 14:22
\documentclass[11pt]{article}
\usepackage[utf8]{inputenc}
\usepackage[T1]{fontenc}
\usepackage{fixltx2e}
\usepackage{graphicx}
\usepackage{longtable}
\usepackage{float}
\usepackage{wrapfig}
\usepackage{rotating}
\usepackage[normalem]{ulem}
\usepackage{amsmath}
\usepackage{textcomp}
\usepackage{marvosym}
\usepackage{wasysym}
\usepackage{amssymb}
\usepackage{hyperref}
\tolerance=1000
\setcounter{secnumdepth}{3}
\author{Luke M Perez}
\date{Summer 2016}
\title{Outline International Relations}
\hypersetup{
  pdfkeywords={},
  pdfsubject={},
  pdfcreator={Emacs 24.5.1 (Org mode 8.2.10)}}
\begin{document}

\maketitle


\section{General Concepts}
\label{sec-1}
\subsection{Levels of Analysis}
\label{sec-1-1}
\begin{enumerate}
\item Individual Level
\begin{enumerate}
\item Human Behavior
\begin{enumerate}
\item Classical IR (Carr, Morgenthau, Neibuhr) focused on ``human
nature'' as \emph{the} cause of war. Rejected as reductionist by
Waltz and structural theorists.
\item Structuralism's strength from 1980ff waning in light of
evolutionary psychology, GT, constructivism
\item Renewed interest of individual levels and their interaction
with state and system suggests potential for dynamic models
of IR (IPE, systems theory, etc.) ::need citation::
\end{enumerate}
\item Human Nature
\item Criticisms
\begin{enumerate}
\item Arguments of human nature (cf. Morgenthau, Neibuhr) are
reductionist
\item Individuals are not the essential actors in IR
\end{enumerate}
\end{enumerate}
\item State Level
\begin{enumerate}
\item Domestic politics pushing upward into the system
\item Examples included
\begin{enumerate}
\item Open Economy Politics
\item Neoclassical realism
\end{enumerate}
\end{enumerate}

\item System Level
\begin{enumerate}
\item Anarchy is a material variable, creates incentives and
constraints on state behavior
\item Criticisms
\begin{enumerate}
\item Waltz relies on theoretical reductionism, treating the state
as a microeconomic firm.
\end{enumerate}
\end{enumerate}
\end{enumerate}
\subsection{Agent-Structure Problem}
\label{sec-1-2}
\begin{enumerate}
\item Who influences who, \emph{agents on structure} or \emph{structures on agents}?
\item Rationalists emphasize the agents as those who make the system and institutions
\begin{enumerate}
\item Wagner (2010) suggests the international system is the product
of international bargains between states
\item Milner (199?) raises the possibility that it could be
\emph{rationalism all the way down} such that important concepts,
like sovereignty, thought to be firm are much more malleable.
\end{enumerate}
\item Constructivists stress the constitutive ontology of agents and structures
\begin{enumerate}
\item Agents and structure emerge together
\item Structure shapes agents in ways that are largely imperceptible.
\begin{enumerate}
\item Wendt (1999) on the culture's of anarchy: Hobbesian, Lockean, Kantian
\item Ruggie (1992): Embedded liberalism thesis. Logic of
free-market, global capitalism baked into the system by the
framers of post-war order.
\end{enumerate}
\end{enumerate}
\end{enumerate}
\subsection{Principle-Agent Model}
\label{sec-1-3}
\subsection{Strategic models}
\label{sec-1-4}
\subsubsection{Interests vs. Preferences}
\label{sec-1-4-1}
\begin{enumerate}
\item Not identical
\begin{enumerate}
\item Preferences are \emph{what} individual actors want.
\item Interests are \emph{why} they want.
\end{enumerate}
\item Norms, morality, or interest may drive interests (Wagner 2010;
Frieden 1999 [Lake and Powell])
\begin{enumerate}
\item preferences and the conflict between them are what drive strategy.
\item NB: Hobbes on the causes of war: competition, diffidence, glory
\emph{vs} Thucydides' fear, pride, interest.
\end{enumerate}
\item Interests are \emph{shaped} by the system. 
\begin{enumerate}
\item Finnemore argues international politics is about defining, not
defending, national interests (1996).
\item Constructivism asks why non-like states produce like behavior
and suggest the answer lies in the conditioning.
\item Waltz's dictum that states evolve toward like units suggests
normative processes at play.
\end{enumerate}
\end{enumerate}
\subsection{Institutions}
\label{sec-1-5}
\subsubsection{Rationalists Definitions}
\label{sec-1-5-1}
\begin{enumerate}
\item International regimes
\begin{enumerate}
\item Laws of War
\item International Organizations
\end{enumerate}
\item Institutions as human made constraints and economic models
\begin{enumerate}
\item ``Institutions are the humanly devised constraints that structure
political, economic and social interaction. They consist of both
informal constraints\ldots{} and formal rules. \ldots{} Together with the
standard constraints of economics they define the choice set and
therefore determine transaction and production costs and hence
the profitability and feasibility of engaging in economic
activity'' (North 1991).
\end{enumerate}
\item Actors (states, non-states) behave in predictable patterns and seek
utility maximizing strategies for any given strategy space (Lake
and Powell 1999).
\end{enumerate}


\subsubsection{Normative Definitions}
\label{sec-1-5-2}
\begin{enumerate}
\item The rules and patterns of behavior Keohane (1987).
\item Cultures of anarchy and norm dynamics
\begin{enumerate}
\item Multiple ``cultures'' of enmity, competition, friendship that form
a path dependency between any two (or groups) of nations (Wendt 1999)
\item Change within and between cultures depends on entrepreneurs who
bring about change in state behavior, ultimately changing the
path dependency of relationships between actors (Finnemore and Sikkink).
\end{enumerate}
\item 
\end{enumerate}
\subsection{Cooperation}
\label{sec-1-6}
\subsubsection{Cooperation \emph{vs} Anarchy}
\label{sec-1-6-1}
\begin{enumerate}
\item Anarchy frustrates cooperation because states are preoccupied with
security (Waltz)
\begin{enumerate}
\item Logic of security dilemma (Jervis 1976)
\item The system incentivizes autarky (Mearshimer)
\end{enumerate}
\item Anarchy \emph{predicts} cooperation because self-help suggests
outsourcing what cannot be accomplished internally (Keohane, etc.)
\begin{enumerate}
\item Anarchy as culture and \emph{the meaning} between two states (Wendt 1992)
\end{enumerate}
\item Anarchy is called into question because cooperation suggests
hierarchy and order and not Hobbesian system.
\begin{enumerate}
\item Milner and the appearance of order (????)
\item Cooperation is a bargaining game (Schelling 1960) and it may be
within a state's interest to cooperate.
\end{enumerate}
\end{enumerate}
\subsubsection{Cooperation and state behavior}
\label{sec-1-6-2}
\begin{enumerate}
\item Harmony and Discord require no change in behavior on the part of actors.
\item Cooperation is \emph{contingent} change in behavior interdependent on
the actions of other partners in the deal.
\end{enumerate}
\subsection{Audience Costs}
\label{sec-1-7}
\subsubsection{Theory (Fearon 1994)}
\label{sec-1-7-1}
\subsubsection{Criticisms}
\label{sec-1-7-2}
\begin{itemize}
\item Limited Scope
\label{sec-1-7-2-1}
\begin{enumerate}
\item The relative strength of the ``changed circumstances'' appeal calls
into question the scope of conditions when audience cost theory holds
\item i.e., if a leader can escape punishment by same ``oh, it was prudent
to raise stakes when I said it, but imprudent to carry out the
threat'' then we might being to wonder if audience costs has any
meaning.
\end{enumerate}
\item Empirical challenges:
\label{sec-1-7-2-2}
Snyder and Borghard 2011 find four points of concern: 

\begin{enumerate}
\item Leaders prefer flexibility in crisis and are therefore more likely
to prefer ambiguity.
\item Domestic public will care more about the substance of the final
policy more than whatever perceived consistency
\item The public concern with the national honor is largely independent
of whatever threats were made.
\item Authoritarian regimes interpret the dynamics of audience costs
differently than democracies, thereby weakening the strength of
audience costs in practice
\end{enumerate}
\end{itemize}

\section{International Political Economy}
\label{sec-2}
\subsection{OEP}
\label{sec-2-1}
\subsubsection{Method and approach}
\label{sec-2-1-1}
\subsubsection{Key findings}
\label{sec-2-1-2}
\subsubsection{Criticisms}
\label{sec-2-1-3}
\begin{itemize}
\item Oatley 2011.
\label{sec-2-1-3-1}
Methodological reductionism produces inaccurate knowledge. Most OEP
seems to drop the final step (model the system with necessary) by
assuming rather than showing that the system does not have an effect.
\end{itemize}
\section{International Organization}
\label{sec-3}

\section{Foreign Policy}
\label{sec-4}
\subsection{History vs Social Science}
\label{sec-4-1}
\begin{enumerate}
\item Three major differences between IR and Diplomatic History (Dip-hist)
\begin{enumerate}
\item Chronology (history) vs Causal mechanisms (IR)
\item Individual events (history) vs Comparative cases (IR)
\item Morality: history more comfortable, IR emphasizes facts over values
\end{enumerate}
\item IR can, and should, draw from history as it builds theories and
hypotheses without falling into an inductive-qualitative trap.
\end{enumerate}
\subsection{Small group dynamics}
\label{sec-4-2}
\begin{enumerate}
\item How do groups perceive another actors behavior (Jervis 1978)
\end{enumerate}



\section{Practice outlines:}
\label{sec-5}
\subsection{IR Fall 2015:}
\label{sec-5-1}
Suppose you are putting together a syllabus for a graduate seminar
providing a survey of the field of international relations to
Ph.D. students who expect to take comprehensive exams in political
science. First, how would you go about organizing your syllabus and
why (e.g. according to research questions? theoretical frameworks?
Research approach- es? Chronologically? etc.)? Second, what are your
goals for what the students should take away from the course?
Explain. Third, what are some alternatives to the answers you've given
to the first two questions, and why would you not adopt those
alternatives?

\subsection{Answer}
\label{sec-5-2}

\subsubsection{How would I approach the question:}
\label{sec-5-2-1}
\begin{enumerate}
\item Organize around levels of analysis
\begin{enumerate}
\item Include discussion of major themes within the levels of analysis
\item Focus on problem driven debates within the field
\item Read works around questions chronologically so as to strengthen
students understanding of theoretical and empirical development
to major questions
\end{enumerate}
\item Why around the levels of analysis?
\begin{enumerate}
\item Unit of analysis is largely an unspoken methodological hunch on
what variable best answers a specific question.
\item Is it always purely pragmatic (cf. Fearon and Wendt 1992; Kydd 2008)
\item -isms are bad and produce pathologies of analysis (Lake 2011; 2013)
\end{enumerate}
\item First image
\begin{enumerate}
\item Political psychology of decision making (Jervis 1976; 1978)
\item Johnson (1974) on presidential leadership styles
\item Rationalism all the way down (Allison and Zelikow 1999) suggests
looking at utility maximizing of state agents
\item Preference formation of citizens (Braumoeller 2010, Scheve and Slaughter?)
\item Democratic theory and audience costs (Fearon 1995): i.e., are
the foreign policy decision makers really bound by the
commitments they make (Schelling 1960)?
\end{enumerate}
\item Second image
\begin{enumerate}
\item Defining of state interests (Finnemore 1996)
\begin{enumerate}
\item arguable that this is a systemic question, but our main
question is how states respond and then then legitimate their interests
\item Frieden (1999): interests are deduced or assumed (when necessary)
\item Frieden and Martin (2001) on the domestic-international
interaction. I.e., that domestic institutions, electoral
design, and other such factors affect interest aggregation
\item Relaxing the assumption of the state as a unitary actor
(Milner 1998; cf. Kydd 2008)
\end{enumerate}
\item Open Economy Politics
\begin{enumerate}
\item Openness as dependent variable; politics as IV (Lake 2009)
\item How much room do states really have (Mosely 2000,2005,2007)
\begin{enumerate}
\item Influence of global finance markets strong, but narrow
because fund managers look to industry-wide metrics
\item RTB logic is flawed but persists because of ideology:
critiques and champions of global capitalism have interest
in narrative that the state cannot, or should not, be the
legitimate arbiter of value
\item In some areas, like FDI, a counter logic of climb to the
top results because best-practice transfer, firms as
advocates for rights, and domestic interests want protections
\end{enumerate}
\end{enumerate}
\item Domestic causes of war
\begin{enumerate}
\item Democratic Theory
\begin{enumerate}
\item Oneal and Russett (1997) on interdemocratic peace and the
challenges by Gartzke () and McDonald (2015)
\item Bennet and Stam suggest that research design could affect
outcome of findings, but some dv/iv relationships, like
democracy, are relatively consistent (2000).
\end{enumerate}
\item Snyder (1991) and the myths of expansionism (echoes
perception problem highlighted by Jervis)
\item War as commitment problem (Powell 2006) and how states seem
to solve it.
\end{enumerate}
\end{enumerate}
\item Third image
\begin{enumerate}
\item Democratic theory
\item Interstate bargaining
\item Questions of anarchy
\begin{enumerate}
\item Materialism of neorealism/neoliberalism
\begin{enumerate}
\item Waltz and the TIP
\item Order within anarchy (Morrow)
\end{enumerate}
\item Constructivism and the agent-structure question
\begin{enumerate}
\item Wendt (1992, 1999)
\item Ruggie (1998?)
\end{enumerate}
\end{enumerate}
\item IPE and the pressures on states
\begin{enumerate}
\item Mosley (RTB vs CTT)
\end{enumerate}
\end{enumerate}
\end{enumerate}
\subsubsection{What are the goals for students to take away}
\label{sec-5-2-2}
\begin{enumerate}
\item Research design:
\begin{enumerate}
\item Approach methods as tool box
\item Model similar to ``med school'' wherein rationalist, qualitative,
and empirical do not compete but compliment (Shapario ????).
\end{enumerate}
\item Exposure to contemporary questions within the field to date
\item 
\end{enumerate}
\subsubsection{Alternatives and why not adopt them}
\label{sec-5-2-3}
\begin{enumerate}
\item Debates around paradigms
\begin{enumerate}
\item flawed approach and generally harmful
\item Did signal how much was generally agreed upon (Waever 1998)
\item But theory driven approach to field has everyone talking past
each other
\begin{enumerate}
\item Trying to ``prove'' other theories wrong, rather than answering
deepest questions about international political phenomena
\item Hard to adjudicate which method is appropriate to the
question when grand theory is major fault-line of discourse
\end{enumerate}
\end{enumerate}
\item Major problems and mid-level theorizing
\begin{enumerate}
\item Strongest alternative possible
\item But possible to miss the forest for the trees
\begin{enumerate}
\item e.g. would be democratic peace
\item Seeing debates over its empirical status rather than
phenomenon that penetrates many other research agendas
\item Debate over its causal pathways cuts across levels of
analysis, and we can see the various weakness and strengths
of different approaches better when grouped by level of analysis.
\end{enumerate}
\item Other research agendas situate better within discussion of
levels of analysis
\begin{enumerate}
\item IOs, for instance, arguably best within second image and
behavior of states vis-a-vis their institutions
\item Cooperation, e.g., could cut across both 2nd and 3rd image
but the other issues at play are best treated on a per-image analysis.
\end{enumerate}
\end{enumerate}
\end{enumerate}
% Emacs 24.5.1 (Org mode 8.2.10)
\end{document}
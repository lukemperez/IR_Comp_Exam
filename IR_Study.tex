% Created 2016-06-30 Thu 14:56
\documentclass[11pt]{article}
\usepackage[utf8]{inputenc}
\usepackage[T1]{fontenc}
\usepackage{fixltx2e}
\usepackage{graphicx}
\usepackage{longtable}
\usepackage{float}
\usepackage{wrapfig}
\usepackage{rotating}
\usepackage[normalem]{ulem}
\usepackage{amsmath}
\usepackage{textcomp}
\usepackage{marvosym}
\usepackage{wasysym}
\usepackage{amssymb}
\usepackage{hyperref}
\tolerance=1000
\setcounter{secnumdepth}{3}
\author{Luke M Perez}
\date{Summer 2016}
\title{International Relations Prelim Exam Notes}
\hypersetup{
  pdfkeywords={},
  pdfsubject={},
  pdfcreator={Emacs 24.5.1 (Org mode 8.2.10)}}
\begin{document}

\maketitle



\section{Outline of International Relations}
\label{sec-1}
\subsection{General Concepts}
\label{sec-1-1}
\begin{enumerate}
\item Levels of Analysis
\label{sec-1-1-1}
\begin{enumerate}
\item Individual Level
\begin{enumerate}
\item Human Behavior
\begin{enumerate}
\item Classical IR (Carr, Morgenthau, Neibuhr) focused on ``human
nature'' as \emph{the} cause of war. Rejected as reductionist by
Waltz and structural theorists.
\item Structuralism's strength from 1980ff waning in light of
evolutionary psychology, GT, constructivism
\item Renewed interest of individual levels and their interaction
with state and system suggests potential for dynamic models
of IR (IPE, systems theory, etc.) ::need citation::
\end{enumerate}
\item Human Nature
\item Criticisms
\begin{enumerate}
\item Arguments of human nature (cf. Morgenthau, Neibuhr) are
reductionist
\item Individuals are not the essential actors in IR
\end{enumerate}
\end{enumerate}
\item State Level
\begin{enumerate}
\item Domestic politics pushing upward into the system
\item Examples included
\begin{enumerate}
\item Open Economy Politics
\item Neoclassical realism
\end{enumerate}
\end{enumerate}

\item System Level
\begin{enumerate}
\item Anarchy is a material variable, creates incentives and
constraints on state behavior
\item Criticisms
\begin{enumerate}
\item Waltz relies on theoretical reductionism, treating the state
as a microeconomic firm.
\end{enumerate}
\end{enumerate}
\end{enumerate}
\item Agent-Structure Problem
\label{sec-1-1-2}
\begin{itemize}
\item Statement of problem
\label{sec-1-1-2-1}
\begin{enumerate}
\item Who influences who, \emph{agents on structure} or \emph{structures on agents}?
\begin{enumerate}
\item Constitutive Theorizing: mutually dependent ontology (constructivism).
\begin{enumerate}
\item Wendt 1999
\item Kowert and Legro 2006: possible to separate as matter of
theory and empirics across time, context, or process.
\end{enumerate}
\end{enumerate}
\end{enumerate}
\end{itemize}
\item Principle-Agent Model
\label{sec-1-1-3}
\item Strategic models
\label{sec-1-1-4}
\begin{itemize}
\item Interests vs. Preferences
\label{sec-1-1-4-1}
\begin{enumerate}
\item Not identical
\begin{enumerate}
\item Preferences are \emph{what} individual actors want.
\item Interests are \emph{why} they want.
\end{enumerate}
\item Norms, morality, or interest may drive interests (Wagner 2010;
Frieden 1999 [Lake and Powell])
\begin{enumerate}
\item preferences and the conflict between them are what drive strategy.
\item NB: Hobbes on the causes of war: competition, diffidence, glory
\emph{vs} Thucydides' fear, pride, interest.
\end{enumerate}
\end{enumerate}
\end{itemize}
\item Institutions
\label{sec-1-1-5}
\begin{itemize}
\item Rationalists Definitions
\label{sec-1-1-5-1}
\begin{enumerate}
\item International regimes
\begin{enumerate}
\item Laws of War
\item International Organizations
\end{enumerate}
\item Institutions as human made constraints and economic models
\begin{enumerate}
\item ``Institutions are the humanly devised constraints that structure
political, economic and social interaction. They consist of both
informal constraints\ldots{} and formal rules. \ldots{} Together with the
standard constraints of economics they define the choice set and
therefore determine transaction and production costs and hence
the profitability and feasibility of engaging in economic
activity'' (North 1991).
\end{enumerate}
\item Actors (states, non-states) behave in predictable patterns and seek
utility maximizing strategies for any given strategy space (Lake
and Powell 1999).
\end{enumerate}


\item Normative Definitions
\label{sec-1-1-5-2}
\begin{enumerate}
\item The rules and patterns of behavior Keohane (1987).
\item Cultures of anarchy and norm dynamics
\begin{enumerate}
\item Multiple ``cultures'' of enmity, competition, friendship that form
a path dependency between any two (or groups) of nations (Wendt 1999)
\item Change within and between cultures depends on entrepreneurs who
bring about change in state behavior, ultimately changing the
path dependency of relationships between actors (Finnemore and Sikkink).
\end{enumerate}
\item 
\end{enumerate}
\end{itemize}
\item Cooperation
\label{sec-1-1-6}
\begin{itemize}
\item Cooperation \emph{vs} Anarchy
\label{sec-1-1-6-1}
\begin{enumerate}
\item Anarchy frustrates cooperation because states are preoccupied with
security (Waltz, Mearshimer, etc.)
\item Anarchy \emph{predicts} cooperation because self-help suggests
outsourcing what cannot be accomplished internally (Keohane, etc.)
\item Anarchy is called into question because cooperation suggests
hierarchy and order and not Hobbesian system.
\end{enumerate}
\item Cooperation and state behavior
\label{sec-1-1-6-2}
\begin{enumerate}
\item Harmony and Discord require no change in behavior on the part of actors.
\item Cooperation is \emph{contingent} change in behavior interdependent on
the actions of other partners in the deal.
\end{enumerate}
\end{itemize}
\end{enumerate}

\subsection{War and conflict}
\label{sec-1-2}
\begin{enumerate}
\item Why War?
\label{sec-1-2-1}
\begin{itemize}
\item Misreading of capabilities.
\label{sec-1-2-1-1}
\end{itemize}
\end{enumerate}
\subsection{International Political Economy}
\label{sec-1-3}
\begin{enumerate}
\item OEP
\label{sec-1-3-1}
\begin{itemize}
\item Method and approach
\label{sec-1-3-1-1}
\item Key findings
\label{sec-1-3-1-2}
\item Criticisms
\label{sec-1-3-1-3}
\begin{itemize}
\item Oatley 2011.
\label{sec-1-3-1-3-1}
Methodological reductionism produces inaccurate knowledge. Most OEP
seems to drop the final step (model the system with necessary) by
assuming rather than showing that the system does not have an effect.
\end{itemize}
\end{itemize}
\end{enumerate}
\subsection{International Organization}
\label{sec-1-4}

\subsection{Foreign Policy}
\label{sec-1-5}

\section{Annotated Readings}
\label{sec-2}
\subsection{Blainey, Geoffrey, (1988) [GB88]}
\label{sec-2-1}
Blainey's \emph{The Causes of War} surveys every major war from 1700 to
roughly 1970, showing how much of the conventional wisdom about the
causes of war are misguided or outright false. War occurs because of
power imbalances and misperceptions about any given nation's position
in the world. Although Blainey lacks the rigor of formal or empirical
models, his findings approach conclusions found by political
scientists: that war is bargaining problem of sorts, a part of the
political process between nations. One notable line of inquiry is his
study of the Manchester creed of the late-19th and early-20th
century. Cooperation in this period was at least as deep and broad as
modern globalization, then as now, the conventional wisdom was that
economics and cultural openness were making war obsolete. But then, as
now, politics rather than economics proves to be sovereign.  
\subsection{Bennett, Andrew (2013) [Bennett2013]}
\label{sec-2-2}
\begin{enumerate}
\item Summary
\label{sec-2-2-1}
\end{enumerate}

\subsection{Broaumoeller, Bear F. (2012) [Broemoeller2012]}
\label{sec-2-3}
\subsection{de Marchi, Scott (deMarchi2005)}
\label{sec-2-4}
Examines the limitations of quantitative and and formal models in
political science, arguing that computational models can compliment
and improve traditional empirical and formal research designs.
Following Achen (2002), de Marchi argues that empirical modeling too
often includes an over abundance of variables, thereby overfitting
their model and producing spurious findings (p. 11). Formal theorists
are not immune to this problem because an abundance of logically
consistent models exist that can be fitted to the empirical
observations such that whatever the researcher wishes to show, \emph{a
priori} can usually be shown. Computation models, however, can
adjudicate between various models by allowing researchers to test a
model against a closer approximation of the Data Generating Process
(DGP) before doing so on the actual empirical data. This approach,
along with \emph{out-of-sample} (OOS) testing helps researchers avoid the
pitfalls of overfitting or underfitting their models.   
\subsection{Oatley, Thomas (2012) [TO12reduct]}
\label{sec-2-5}
Oatley critiques the methodological reductionism of OEP because it
risks producing false or inaccurate knowledge. According to Oatley,
OEP assumes---rather than shows---that the system under study can be
studied without consideration of system level effects. In at least
three issue areas he shows that modeling the system level effects
produces different findings from a strict OEP method that only models
domestic level variables. 
\subsection{Waltz, Kenneth N. 1959}
\label{sec-2-6}
\emph{Man, the State, and War} from Waltz's dissertation, examines the
levels of analysis (individual, state, system) and the causes of war
and peace among nations. 
\subsection{Waltz, Kennith N. 1979}
\label{sec-2-7}
\emph{Theory of Internat'l Politics} lays the foundation for nearly
all contemporary IR research either by critiquing, extending, or
modifying Waltz's basic definitions of theory, reductionist/systemic
approaches, and philosophy of social science. Waltz's microeconomic
method and systemic approach recast classical realism into it's
neorealist, or structural, formulation found in Mearshimer and others.
\begin{enumerate}
\item Systems Analysis, when it's required (p. 57)
\label{sec-2-7-1}
Waltz concludes form his analysis of Kaplan that a systemic approach
should be used only if it ``seems'' that system level causes are at
play. Although this a is somewhat useful criterion, he leaves
undefined what exactly he means by \emph{seems}.
\item Statistics as mere description (p. 3)
\label{sec-2-7-2}
``Statistics are simply description in numerical form'' (p. 3). 

Note here that Waltz is writing in 1979, just before the cusp of the
computational revolution in social science. He may only be referring
to descriptive statistics and not the more advanced methods used by
contemporary scholars of IR.
\end{enumerate}
% Emacs 24.5.1 (Org mode 8.2.10)
\end{document}